\begin{flushleft}

\chapter{GM Guide: How to GM} \label{gmguide}

Being a GM is both incredibly hard and incredibly fun. You are in control of
the world around your players, and your players learn about and interact with
the world through you.

You are responsible for letting the players know what their characters see and
know about about the world around them, and how their actions and the actions
of the NPCs affect them. Here are a few examples of things you might let the
players know.

``Dr. Eeevil appears to be brewing a strange concoction"

``After Dr. Eeevil threw the concoction at your feet, you start to feel really
tired".

At times, it might be necessary to have the player roll their dice to see what
happens. When this happens, determine the difficulty of the task and tell the
player what their target number is (default of 5). Remember, if you want to
have something happen, there is a chance the rolling might cause that not
to happen. In these cases, it might be better not to give the opportunity to
make that roll. Bypassing a potential roll for enjoyment and story progression
is usually better than setting a super high threshold.

To follow up on the previous examples, the player wants to try blow the
concoction at his feet towards Dr. Eeevil.

``You take a deep breath to use your super breath, but you inhale more of the
concotion's fumes. Roll your die to try resist the effect of the potion. Your
target number is 6."

Getting information from your players is just as important as giving your
players information on their surroundings. Playing a tabletop game is all
about playing out the actions the character would take, and a large part of
that is player choice.

Player choice is often a double edged sword, though. With the ability to make
choices freely, the players can and will often make a choice you could not have
guessed they would do. This often leaves your plan in ruins. This is perfectly
normal, and after the first few times it happens, you will get better at
modifying your plans when you need.

GM: ``As your friend falls face first on the floor into Dr. Eeevil's
concotion, what do you do?"

Player ``I want to use my power that gives me resistance to poisons and take a
drink. Then I'll give Dr. Eeevil a thumbs up and spit some in his face."

GM: ``Well, roll that attempt to spit the concotion in his face and let me
reread my notes on it quick."

\addcontentsline{toc}{section}{Setting up the game}
\section*{Before the Game}

Before playing the game, it's a good idea to talk with your players and find out
what things are ok and not ok to have in the game. Some players may be
uncomfortable with adult, political, religous, or other topics and not want to
see them in the game at all. Alternatively, they may be uncomfortable with a
subject, but are ok with having it in small quantities in order to try learn
more about it. Knowing this beforehand is important to ensuring everyone is
comfortable and having fun.

In addition to finding out what your players want in the game, it's important
to find out what kind of game your players want to play. It's unfortunate when
the GM spends a lot of time on preparing a game and no one playing the game
is interested in playing the game.

PRPG can be any kind of game the players are wanting to play. For example, do
the players want to play a mystery game in a steam punk setting? If so, come up
with a city to start the players in and give them a mystery to start on.
If a setting and world is already available that the players want to play in,
that could be used too. If you need help coming up with a world to play in,
see chapter 6.

Make sure to talk about the characters your players are going to play as well.
Understanding the motivations the characters have will allow you to present
interesting choices to your players.

\addcontentsline{toc}{section}{Types of players}
\section*{Types of Players}

Knowing what kind of players you have is also very important. This lets you
create things for specific players to do. You'll find that many players, very
broadly speaking, fall into one of the following categories more than another.

Method Actors: Players who enjoy acting out exactly how their character would
react to a give situation. Give them something their character finds
interesting, whether it's choices, an item, or NPCs to interact with, and they
will have a great time.

Tacticians: Players who enjoy coming up with a strategy and seeing it executed
to perfection. Unfortunately, PRPG is not aimed entirely at tacticians. It is
still possible to provide situations for them to plan through. When doing so,
provide them something concrete to plan with. For example, a map of a building,
an artifact that has a particular use, etc.

Story Builders: Story builders are along to tell an interesting and engaging
story with their character. Oftentimes they want to have story arcs that
go through with a fulfilling (or cliff hanging) conclusion and some good
character development along the way. They enjoy both looking back and seeing how
their character affected the story, and thinking about what things their
character may end up doing next. While giving all players choices that impact
the story is a Good Thing, take care to ensure the Story Builder gets to make
some choices.

\addcontentsline{toc}{section}{Setting up the game}
\section*{Setting up the game}

Planning for and setting up a game can be tricky if you have never done it
before. Thankfully, with PRPG, you have the freedom to plan any game very
easily.

If your players have a game type they want to play, center your planning around
that. For example, it doesn't make sense to plan a dragon attack on a village
when your players want to explore the ancient mysteries of a long dead alien
race on another planet. This way you plan for something your players want to
interact with.

\addcontentsline{toc}{section}{Running the game}
\section*{Running the game}

The GM is responsible for letting the players know what their characters see
and, if applicable, know about certain subjects. It's always acceptable to
have the players roll for something if you're not entirely sure. If they roll
really well, maybe they do know something about what's going on.

In addition to letting the players know what their characters see, the GM is
responsible for letting the characters know how the world around them reacts
to the things they do. Does that blinged out hippy living in a spare
compartment on a blimp react well to the character's attempts to rent out his
house? It's up to you to find out and let the players know once you do.

It's possible to accomplish figuring out what happens with rolls, both by the
players and the GM. If the player needs to roll, tell them the threshold they
need to beat. The following table shows what various difficulty levels are
in relation to a threshold modifier.

\label{threshold_difficulties}
\begin{center}
\begin{tabular}{ |c|c|}
\hline
easy & -3, -2, -1 \\
\hline
average & -0 \\
\hline
difficult & +1, +2 \\
\hline
very difficult & +3 \\
\hline
nigh impossible & +4, +5 \\
\hline
\end{tabular}
\end{center}

\end{flushleft}
