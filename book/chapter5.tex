\chapter{GM Guide: Creating worlds}

Coming up with a world from scratch can be very difficult. This chapter provides
examples of several worlds you could use and walks through creating a world
from scratch. If you already know what the world you are playing in is like,
just skip this chapter.

To create a world quickly when the players want to start a game, answer the
following questions.

\begin{enumerate}

    \item{What setting is the world?}
    \item{What is the world's name?}
    \item{What country are the characters starting in?}
    \item{What city are the characters starting in?}
    \item{What brought the characters together?}

\end{enumerate}

Most of these are very simple to do. Naming the city, country, and world should
be something that takes a few seconds. Likewise, finding out how the players
ended up together shouldn't take too long. You can always have the players
talk through how they ended up together while you work on the world.

The setting and technology level are a little more important to the game and
are often linked together. Once again you can always ask your players what kind
of game they want to play and have them discuss and decide on a setting.

\addcontentsline{toc}{section}{Setting}
\section*{Setting}

The setting sets the overall culture and mood of the game. For example,
a dystopian steam punk setting should feel different than a medieval fantasy
setting. Here is a list of settings to spark an idea.

\begin{enumerate}

    \item{alien invasion}
    \item{cultural renaissance}
    \item{cyberpunk}
    \item{high fantasy}
    \item{horror}
    \item{medieval fantasy}
    \item{post apocalyptic}
    \item{space fantasy}
    \item{steampunk}
    \item{superhero}
    \item{victorian era}

\end{enumerate}

If you have trouble coming up with an idea, roll a die and choose that number
from the list. Remember to subtract one and reroll if you had a 6.

The setting generally helps set the general technology level of the world.
Having the players start in a city with a technology level that makes sense
in the setting is a good idea because it lets them explore the setting they
wanted to play in. It's always possible to introduce higher or lower tech
civilizations during the game if needed.

\addcontentsline{toc}{section}{Example world creation}
\section*{Example world creation}

It's 17:00, your friends arrived, they're ready to play, but you haven't made
the world yet. How will you ever survive?

You think back to talking with your friends and remember them mentioning the
word ``steampunk" a lot when you were seeing if they were interested in playing
a game. Unsure what steampunk is, you do a quick goole search and find out that
steampunk is a genre defined by incorporating technology and aesthetic designs
inspired by 19th century industrial steam-powered machinery. While not a
complete description, you feel confident that you can throw in a lot of steam
powered tech into the game.

You decide to name the world Challac for no particular reason, and name the
country the players are in Zigwan. You see one of your players boiling some
water for tea and decide to call the city Hydronia. You call over to your
players ``Alright, so how do you all know each other?"

You listen while they discuss this for a few minutes. It sounds like a large
dimensional portal showed up in their house one day. They grabbed a few things
they though was important and then hopped through the portal. Now they've all
ended up in the city of Hydronia. You decide they're in the boiler disctrict.

``Alright", you say, ``After you all jump through the portal, you see a large
flash of light. You all feel like you're falling both up and down at the same
time, but you're not moving. You see a faint light on the other end, and
suddenly, without warning, you're thrust out into an unfamiliar place. If you
want to try land without falling over, make a roll"

And the game begins!
