\begin{flushleft}

\chapter{Dice rolling and thresholds} \label{dice_mechanics}

When playing the game, situations will arise where the players want to
accomplish something specific. At these points, you need to understand the dice
rolling rules and thresholds. Dice rolling is discussed next, and thresholds
immediately after that.

\addcontentsline{toc}{section}{Rolling Dice}
\section*{Rolling Dice} \label{dice_rolling}

There are 3 different situations to watch for when rolling a die.

\begin{enumerate}
    \item{The die can show a 1}
    \item{The die can show any number from 2-5}
    \item{The die can show a 6}
\end{enumerate}

If the die shows a 1, that means something bad could possibly happen. The GM
gets to choose whether or not to have something bad happen. If something does
happen, it should be something that causes some sort of inconvenience, minor or
large, to the player who rolled the 1. Just remember that having major problems
everytime a 1 is rolled isn't very fun.

If the die is a 2, 3, 4, or 5, the player checks if the result is equal to or
higher than the threshold (discussed in the next section). If the result is
equal to or higher than the threshold, the action succeeds. If the result is
lower than the threshold, then the action fails. The higher the dice roll is
above the threshold, the easier the character makes the action seem to be.

If the die shows a 6, the player subtracts 1 from the roll (to make it 5)
and rolls the die again. The second roll is added to the first dice roll. Note
that if the die shows a 1 on the second roll, the result is added to the roll
without any bad stuff happening.

\begin{enumerate}
    \item{Roll one die}
    \item{If the die shows a six, subtract 1 and roll the die again}
    \item{Add the result of the second roll to the first}
\end{enumerate}

If the second roll also shows a 6, the action succeeds in a spectacular way
provided it is possible to do. If the GM wants to allow it, this could also
mean that ANY action, no matter how improbable to succeed, would succeed if
both dice show a 6. GM discretion is heavily advised here though.

\addcontentsline{toc}{section}{Thresholds}
\section*{Thresholds} \label{dice_thresholds}

A threshold is a number that a roll must meet or exceed to be considered
successful. The base threshold for any task is a 5, meaning  the dice must show
a 5 or a 6. When writing out thresholds, the following format will be followed.

\begin{quotation}
\centering
    skill or action (threshold)
\end{quotation}

For example, kicking down a door with a threshold of 6 would be

\begin{quotation}
\centering
    kick down a door (6)
\end{quotation}

Examples of how difficult a threshold is considered are provided in chapter
\ref{gmguide} on page \pageref{threshold_difficulties}.

The threshold can be modified by the GM depending on how difficult the GM
believes the task would be for the chracter. This modification can be anywhere
from -5 to +5 making the easiest threshold 0 (no roll, automatic success) and
the hardest threshold 10. The player is always free to let the GM know of any
reasons the character might be particularly skilled in an action.

For example, if Kristi wants to distract the guards, the GM decides that
while yes, they would be distracted, it would be very difficult to distract
them as fully as Kristi wants. The GM decides the difficulty is modified by
+2 making the test a singing (7) test.

Kristi rolls her die and gets a 6. Because rolled a 6, and must have higher
than a six to succeed, she subtracts 1 from the roll, making the roll a 5, and
rolls her die again. Her second roll is a 5. She then adds the 5 and the 5
together to get 10. 10 is far above the threshold of 7, so her singing attempt
wows the guards and draws in a small crowd. Some sound technicians visiting
town hand her a microphone and set up an impromptu stage with speakers.

\addcontentsline{toc}{section}{Types of Rolls}
\section*{Types of Rolls} \label{roll_types}

There are two types of rolls: contested and uncontested.

Contested rolls happen when one person is trying to accomplish something while
another person or thing is working against that goal. Contested rolls work by
having both parties roll their die and seeing who has the higher result. In
the event of a tie, nothing happens.

Uncontested rolls are when a character tries to do something that doesn't have
someone or something working to interfere with the result. The roll is against
a threshold defined by the GM (default of 5 if no threshold is specified).

Here are some examples of contested rolls.

\begin{enumerate}

    \item{A character tries to punch a baddie, but the baddie wants to block}
    \item{A character is trying to hide from Dr. Eeevil's Seek'n'destroy bot
          \textsuperscript{TM}}
    \item{A character by is trying to steal a bottle of liquid hydrogen from
          a poor laboratory worker}

\end{enumerate}

Here are some examples of uncontested rolls

\begin{enumerate}

    \item{A character wants to drive a motorcycle with one foot}
    \item{A character is trying to rewire a timebomb}
    \item{A character with an addiction to rare tea has to resist the urge to
          drink the rare and imported tea Dr. Eeevil handed him}

\end{enumerate}


\end{flushleft}
