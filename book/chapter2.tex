\begin{flushleft}

\chapter{Dice rolling and thresholds}

\addcontentsline{toc}{section}{Rolling Dice}
\section*{Rolling Dice}

When playing the game, situations will arise where the players want to
accomplish something specific. For example, Kristi wants to use her character's
Sing a Pop Song power to try to distract some guards. Kristi states what power
she is going to use, rolls her die, and checks if it is above the
threshold of success.

There are 3 different situations to watch for when rolling a die.

\begin{enumerate}
    \item{The die can show a 1}
    \item{The die can show any number from 2-5}
    \item{The die can show a 6}
\end{enumerate}

If the die shows a 1, that means something bad could possibly happen. The
player rolls the die again to see just how bad it is. If the die comes up as a
1 again, something disastrous occurs. If the die is a 2, 3, or 4, something
minorly bad happens. If the die shows a 5 or a 6, the character's action
had no effect without any bad result.

If the die is a 2, 3, 4, or 5, the player checks if the result is higher than
the threshold. If the result is higher than the threshold, the action succeeds.
If the result is lower, then the action failed. The higher the dice roll is
above the threshold, the easier the character makes the action seem to be.

If the die shows a 6, the player can subtract 1 from the roll (to make it 5)
and roll the die again. The second roll is added to the first dice roll. Note
that if the die shows a 1 on the second roll, the result is added to the roll
without any bad stuff happening. If the second roll also shows a 6, the action
succeeds in a spectacular way provided it is possible to do.
If the GM wants to allow it, this could also mean that ANY action, no matter
how improbable to succeed, would succeed if both dice show a 6. GM discretion
is heavily advised here though.

An alternative rolling system you may use is to roll two dice at once, sum the
result, and subtract 1. Be warned, however, that if either die shows a 1,
something bad happens as determined by the other die. For example, if you
roll a 6 and a 1, your character fails to accomplish anything with the attempt.

\addcontentsline{toc}{section}{Thresholds}
\section*{Thresholds}
The base threshold for any task is a 4, meaning  the dice must show a 5 or a 6
when rolled. This threshold can be modified by the gm depending on how
difficult the GM believes the task would be. This modification can be anywhere
from -4 to +5 making the easiest threshold 0 (no roll) and the hardest
threshold 9. The player is always free to let the GM know of any reasons the
character might be particularly skilled in an action.

To make rolling for thresholds above 6 possible, the player follows this
process when making a roll for a higher threshold.

\begin{enumerate}
    \item{Roll one die}
    \item{if the die shows a six, subtract 1 and roll the die again}
    \item{Add the result of the second roll to the first}
\end{enumerate}

For example, when Kristi wanted to distract the guards, the GM decides that
while yes, they would be distracted, it would be very difficult to distract
them as fully as Kristi wants. The GM decides the difficulty is modified by
+3. Kristi rolls her die and gets a 6. Because she needs to be above a six
to succeed, she subtracts 1 from the roll, making the roll a 5, and rolls her
dice again. Her second roll is a 5. She then adds the 5 and the 5 together to
get 10. 10 is far above the threshold of 7, so her singing attempt wows the
guards and draws in a small crowd as some sound technicians visiting town hand
her a microphone set up an impromptu stage with speakers.

\addcontentsline{toc}{section}{Types of Rolls}
\section*{Types of Rolls}

There are two basic types of rolls: contested and uncontested.

Contested rolls happen when one person is trying to accomplish something while
another person or thing is working against that goal. Contested rolls work by
having both parties roll their die and seeing who has the higher result. In
the event of a tie, nothng happens.

Uncontested rolls are when a character tries to do something that doesn't have
someone or something working to interfere with the result. The roll is against
a threshold defined by the GM (default of 4) as usual.

Here are some examples of contested rolls.

\begin{enumerate}

    \item{A character tries to punch a baddie, but the baddie wants to block}
    \item{A character is trying to hide from Mr. Eeevil's Seek'n'destroy bot
          \textsuperscript{TM}}
    \item{A pedestrian walking by is trying to steal something from a
          character}

\end{enumerate}

Here are some examples of uncontested rolls

\begin{enumerate}

    \item{A character wants to drive a motorcycle with one foot}
    \item{A character is trying to rewire a timebomb}
    \item{A character needs to resist the urge to drink all the tea}

\end{enumerate}


\end{flushleft}
