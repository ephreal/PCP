\begin{flushleft}

\chapter{Example Powers} \label{example_powers}

This chapter provides example powers and power costs to help the players and
GM come up with sane powers and power costs. As always, the GM is the final
arbitrator when it comes to power costs.

\section*{Terrible power ideas}

Let's start off with examples of terrible powers to have in the game. Many of
these are terrible because they are either ill-defined or remove any need to
strategize. The reason for each power being a terrible power is given.

\begin{center}
\begin{tabular}{ |p{6cm}|p{6cm}|}
\hline
Do Anything & This power is too broad. \\
\hline
Can pull literally anything out of a bag & Many adventures revolve around
trying to find an item. This would make the adventure un-necessary \\
\hline
Invulnerable & Too broad. \\
\hline
Can transform into a super powered state after nearly dying, then eating a bean,
then avoiding notice for ten minutes, then.... & Too specific and complex. \\
\hline
\end{tabular}
\end{center}

\section*{Powers that require no points to use}

Some powers are small enough that they can be used without any cost. Every
character can start the game with one of these. These should be non-game
breaking powers that the player will use often.

\begin{center}
\begin{tabular}{ |p{6cm}|p{6cm}|}
\hline
Change clothing super fast & Could be useful in some situations \\
\hline
non-powerup transformation & Provides differences in appearance \\
\hline
Higher stamina than most people & The character might be able to run at high
speeds for longer than most \\
\hline
Is magical & Some heros are defined by being different in some magical way \\
\hline
Can change personalities & Mental change, no changes other than personality.
This could even be a required thing to do to use other powers. \\
\hline
Smartest person in the universe & A thing the character has \\
\hline
Enhanced Vision & Small, innocuous, and just a thing the character has \\
\hline
enhanced strength & About as strong as a bear. Seems to be common in shows \\
\hline
\end{tabular}
\end{center}


\section*{Example powers requiring 1 point to use}

Any power that could accomplish something useful should be given a cost to it.
Things that are small and will be used often and repeatedly should be given a
small cost to use.

\begin{center}
\begin{tabular}{ |p{6cm}|p{6cm}|}
\hline
Transformation for small power-up & Common theme in many shows \\
\hline
Perform a trickshot perfectly & Useful and would be used often \\
\hline
Immunity to toxins for 10 minutes & Specific enough and useful \\
\hline
See through an animals eyes for some time & Common and useful. \\
\hline
Limited period of extreme super strength & A common power in animes \\
\hline
Instant Armor & Create a shield out of thin air to block at most 3 damage \\
\hline
\end{tabular}
\end{center}


\section*{Example powers requiring 2 points to use}
Stronger powers should require more ponts to use. In the 2 point range, powers
that are the character's super useful and highly used skills should be
included. There is a little bit of overlap with skills that require 1 point,
the main difference is in how powerful of an effect the skill has.

\begin{center}
\begin{tabular}{ |p{6cm}|p{6cm}|}
\hline
Transformation for strong powerup & Might take a little time to do too \\
\hline
Shoot a beam of death from the eyes & You know who this is referencing \\
\hline
Speed healing 5 damage & Useful and semi common \\
\hline
Fire breath & Ever wanted to be a dragon? \\
\hline
Temporarily nullify a power & Yours, a friend's, an enemy's...\\
\hline
 Shield & Create a shield out of thin air to block at most 3 damage \\
\hline
\end{tabular}
\end{center}

\end{flushleft}
