\begin{flushleft}

\chapter{GM Guide}

Before playing the game, a few things need to be taken care of.

\addcontentsline{toc}{section}{Setting up the game}
\section{Before the Game}

Before creating a character, the players and GM should talk about what kind of
game and setting the players want to play.

The GM and players should have a brief talk about what things the players
would like to see in the game and what things they would like left out.
Some players may be uncomfortable with adult, political, religous, or other
topics and not want to see them in the game at all. Knowing this beforehand
is important to ensuring everyone is having fun.

PCP can be any kind of game the players are wanting to play. For example, do
the players want to play a mystery game in a steam punk setting? If so, come up
with a city to start the players in and give them a mystery to start on.
If a setting and world is already available that the players want to play in,
that could be used too.

\addcontentsline{toc}{section}{Running the game}
\section{Running the game}

The GM is responsible for letting the players know what their characters see
and, if applicable, know about certain subjects. It's always acceptable to
have the players roll for something if you're not entirely sure. If they roll
really well, maybe they do know something about what's going on. That said,
don't roll for everything. Some things, like walking, eating, reading, and
other simple tasks are considered automatic and do not require a roll.

In addition to letting the players know what their characters see, the GM is
responsible for letting the characters know how the world around them reacts
to the things they do. Does that blinged out hippy living in a spare
compartment on a blimp react well to the character's attempts to rent out his
house? It's up to you to find out and let the players know once you do.

It's possible to accomplish figuring out what happens with rolls, both by the
players and the GM. If the player needs to roll, tell them the threshold they
need to beat. The following table shows what various difficulty levels are
in relation to a threshold modifier.

\begin{center}
\begin{tabular}{ |c|c|}
\hline
easy & -3, -2, -1 \\
\hline
average & -0 \\
\hline
difficult & +1, +2 \\
\hline
very difficult & +3 \\
\hline
nigh impossible & +4, +5 \\
\hline
\end{tabular}
\end{center}

\end{flushleft}
