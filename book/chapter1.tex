\begin{flushleft}

\chapter{Introduction}

Welcome to Pop Culture Protagonists (PCP), a simple tabletop RPG system designed
for ease of character creation, ease of play, and ease in general. The system
was designed for people who have never played a tabletop RPG before and may
want to play but have been put off by complicated rulesets before. In PCP,
rules are kept to a bare minumum to provide simple guidance on how to resolve
things, giving both the person running the game (the GM) and the players great
flexability in what they want to try doing.

PCP allows for the creation of whatever the player would like to play, limited
only by player imagination and GM discretion. The intent is to allow a player
to take a character from pop culture and create them in a few minutes.

Maybe you want to create someone who summons monsters by placing down cards on a
device attached to your wrist? Or maybe you want someone who has a
transformation sequence with a wand and moon based powers? Or what about your
favorite sword-bearing, super-reflexes, high-jumping, speedy,
box-on-back-wearing anime character? In PCP, you can do all those and more.

\addcontentsline{toc}{section}{Required Materials}
\section*{Required materials}

PCP requires the following to adequately play it

\begin{enumerate}
    \item Six Sided Dice
    \item Some paper or a computer
\end{enumerate}

PCP uses six sided dice because most people have a six sided dice around
their house somewhere. It's possible to play the game with one or two dice.
Depending on the rolling method chosen, using 2 dice may be faster.

Having a way to record your character's current stats is a good idea as well.
Using a piece of paper or a computer with a text editor like notepad open is
a fairly easy way to do this. In general, there are 4 things to track: focus,
health, protagonist points, and powers. A large amount of paper shouldn't be
needed.

Finally, in addition to some six sided dice and paper, it's also highly
recommended to bring a good sense of humor to the game, as zany antics are very
likely to happen with the loose ruleset in the game. That doesn't mean serious
games can't happen, just that they are far more likely to move towards the
inane.

\end{flushleft}
