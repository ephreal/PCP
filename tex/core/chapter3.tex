\begin{flushleft}

\chapter{Character Creation} \label{character_creation}

Character creation consists of three steps.

\begin{enumerate}

    \item{Come up with or find a character concept}
    \item{Come up with powers for that character}
    \item{Get the powers checked by the GM}

\end{enumerate}

\addcontentsline{toc}{section}{Character concepts}
\section*{Coming up with a character concept}

While it's entirely possible to come up with a character concept out of thin
air, it's recommended to use a character from a show or movie you like. This
makes it easier for you to have a personality and a power set in mind.
Hopefully, the GM will also be somewhat familiar with the character. If
not, a quick google search should give some idea of what to expect.


\addcontentsline{toc}{section}{What makes up a character?}
\section*{What makes up a character?}

In PRPG, a character has 4 things that need to be tracked. Health, Protagonist
Points, Powers, and Focus.

Health is a measure of how much damage the character can take and how badly
injured the character looks. Every character starts with 10 points which can be
increased later on with 2 focus points for 1 health. When the character
reaches 0 health or bleow, the character is unconscious. If the character is
still taking damage and falls to -10 health, he or she dies.

A character's Protagonist Points are a measure of how many times the character
can use powers. Because powers are central to the game, every character starts
with 20 protagonist points. The points are all recovered anytime the character
has had a chance to get a good rest in (sleep or otherwise). Points can also be
recovered anytime the character acts particularly `protagonist-y' (GM's
discretion). The amount of points a character has can be increased by 4 focus
points for 1 power point.

Powers are things the character is able to do that the average person cannot.
Most of these powers cost points to use, although each character will have
one that does not. The one that requires no points is assumed to be the
character's most commonly used power that does not have a major impact on the
character and story, for example, super strength. Because powers are made by
the players, the GM should double check them to ensure nothing is too powerful
(according to the GM). Powers can be gained by spending focus later on. The GM
decides how much a particular power costs to use. More powerful powers should
cost more points to use.

Focus represents what the character spends their time improving. Points are
gained for particularly `protagonist-y' acts, accomplishing tasks, and by
the GM at the end of any game. Charaters start out with no focus points. Focus
is mostly important for games that will extend beyond more than one session.

\addcontentsline{toc}{section}{Coming up with powers}
\section*{Coming up with powers}

Each character is able to start the game with 4 powers: 1 innate power that
either costs no points or is always on, and 3 that cost protagonist points to
use. The player comes up with powers that fit the character, and the GM
decides how many points the power should cost. Avoid horribly complicated and
detailed powers.

\addcontentsline{toc}{section}{GM Review}
\section*{GM Review}
It's important that the GM at least glance over the powers to ensure they are
a good match within the game planned and the power cost is adequate. A player
shouldn't have to spend 6 points on a power that lets her summon a small blue
light. Conversely, neither should the player be able to spend 1 point to
summon Galaxar, destroyer of worlds and eater of souls.

\addcontentsline{toc}{section}{Character Creation Example}
\section*{Character Creation Example}

Due to worries of copyright complications, here is a fake show that hopefully
sounds somewhat familiar to you.

ChaWiOh has young, spiky haired protagonist with an ability to transform into
an older mysterious version of himself. The character is able to place cards
onto a machine that extends from his wrist to summon monsters according to many
(often arbitrary and seemingly made up) rules. These monsters vanish once any
combat (dueling) is done.

Let's create the main character from ChaWiOh: Chawi.

Chawi has one notable power that should require no points to use: The ability
to transform into an older version of himself and back again. Because this
power does not change anything about Chawi besides physical appearance and
general dueling ability, the GM decides to make this the free to use ability.

In addition to this power, other powers that Chawi seems to have because he is
a main character in the show are

\begin{enumerate}

    \item{Picking the right card at at improbable times}
    \item{Forcing opponents to duel with cards instead of other weapons}
    \item{Summoning monsters by playing cards}

\end{enumerate}

The GM looks over the powers and decides that the first one will cost 1 point,
the second costs 2 points, and the third costs 2 points.

With that, Chawi is ready to play.

\end{flushleft}
