\begin{flushleft}

\chapter{Skills} \label{Skills}

It's common to want the players to have some tangible benefit for their
character's progress as they play the game. Maybe the characters have been
spending a lot of time cooking food and the GM wants their success at cooking
to go up in order to reflect their increased skill at cooking. Of course, you
can always modify thresholds to be lower to reflect skill increase too, but
it's easier to remember to add a skill to the threshold than work out when the
threshold is supposed to be lower.

Because skills give a direct bonus to exceeding the thresholds, it's recommnded
to have skills that are no higher than 3. Anything over 3 will ensure that
the base threshold will always be exceeded.

When the player does an action that is related to the skill, the player rolls
their dice and adds their skill to the result. As always, rolling a 1 is still
a failure and may have something bad happen.

Similarly to thresholds, skills are written as skill(level). That means having
a skill of 3 in fighting is written as fighting[3] with square brackets to
emphasize this is a skill, not a threshold.

\section*{Using skills} \label{using_skills}

Skills provide a direct bonus to rolls the player makes that are related to
that skill. A fighting[3] skill gives the player an additional additional
3 to add to any roll they make related to fighting, while a swimming[1] skill
gives the player an additional 1 to add to any swimming related rolls.

For example, Carl has a 2 in potion making. He rolls his die and gets a 4. He
then adds 2 to the 4 he rolled for a total of 6. Given that the base threshold
is 5, he has successfully made the potion.

\section*{Example skills} \label{example_skills}

The amount of skills that could be used are endless. Here are some skills to
spark some ideas into what skills are.

Alchemy

Armor use

Cooking

Dancing

Dodging

Fighting

Hunting

Negotiation

Persuasion

Pickpocketing

Stealth


\end{flushleft}
